\chapter{总结与展望}

\section{本文工作总结}
在压缩感知理论应用到脉冲相机领域的实现中,本文首次将多窗口特征提取融合引入脉冲压缩感知算法之中,实现了保证模型轻量化的前提下实现了脉冲时空相关性的利用,在基于REDS的模拟数据集下展现出相比其余SOTA脉冲重建算法在多个图像重建指标个推理延迟上的显著优势。

\section{未来研究展望}
脉冲相机作为一种新型的神经形态传感器,业已在交通、军事、工业生产等方向实现应用并取得良好成效。但是脉冲数据本身的密集性,较差的语义包含和暗光条件下的易干扰性都成为其在上述领域应用的不利条件和阻碍其应用到更多领域的重要原因。本节将对该领域所存在的问题和挑战进行讨论,并对可能的研究方向进行展望。

第一:无监督脉冲视觉方法亟需突破。目前的脉冲视觉算法大多数是建立在有监督的脉冲-图像数据集上的。在真实快速运动场景中,捕获真实纹理丰富的图像是较为困难的。现有的高速运动数据集大多是建立在高速相对运动场景下的,将基于其训练的网络延伸至真实高速场景可能会导致较大误差。同时脉冲视觉领域尚处稚嫩阶段,开展大规模数据标注工程量极大以至于现有资源不可实现。因此开展无监督脉冲视觉方法,降低算法对标注的依赖性是当务之急。

第二:脉冲阈值自适应。脉冲相机在获取场景运动信息时,并未对整个场景进行内容判断,也不能根据光照强度自适应的调整发射阈值以避免出现强光下脉冲的过发射和弱光下的欠发射。这对于后续进行场景目标检测,运动追踪等下游任务带来不利影响。在脉冲相机持续拍摄过程中有重点性的对高速场景和静态背景分别进行阈值调整,保证在不同光强下的脉冲一致性,对所有利用脉冲的场景均具有普惠作用。

第三:多模态多视角结合。脉冲相机自身具有相较于传统相机功耗低,时间分辨率高的优势,也存在其空间分辨率低,噪声较高的弱点。针对不同场景,应采用RGB相机,事件相机,红外相机等视觉传感器多角度全方位地获取场景信息,还可以同时依靠听觉,触觉等视觉力不能及的官能在特定场景如水下、强光等视觉受限时保证感知能力。

第四:脉冲相机承担具身智能的感知任务。现有的脉冲相机多用于离线检测,尚未建设成熟的在线感知系统用于对场景实时开展”感知-决策-行动“行为链。而高速运动场景多出现于无人机,自动驾驶,机器人等对实时性要求极高的场景。更好的结合具身智能算法,将自身优势落实到具体任务之中,推向传统相机因自身欠缺而无法深入的领域中去,增进相关方的利益。