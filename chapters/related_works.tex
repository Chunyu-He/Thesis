\chapter{相关工作介绍}

\section{人类感光系统原理}
由于近年来蓬勃发展的神经形态视觉传感器追根溯源是受到了人类感光系统的启发,所以有必要讲解一下人类感光系统的原理,并随后分析这种原理如何被创造性地抽象化,利用和改进,以表现出和传统曝光相机在采样机制和数据结构上迥然不同的特点和成像高动态范围的优势。
\subsection{人类感光系统}
人类感光系统是多种细胞组成不同生理结构的有机结合,其核心为视网膜。如图\ref{fig:human_eye}所示,视网膜是一个由多种细胞组成的结构,是位于人眼眼球内侧的高度复杂的精细感光结构。从外层到内层分布着光感受器细胞外段层,包含视锥细胞和视杆细胞的外核层,外网状层,包含水平细胞和双极细胞的内核层,内网状层和向神经中枢传递神经信号的神经节细胞。正因为这些细胞的层次化有序排列,才能使得人眼拥有高分辨率,高动态范围的特性。这为神经形态视觉传感器提供结构和功能上的借鉴。

视网膜感光层的核心是视锥细胞和视杆细胞,它们将光信号转变为人脑可接受处理的电信号。视锥细胞对强光敏感,负责颜色识别,在中央凹区域尤为密集,也恰好是光线入眼的直接接受区域。视杆细胞对弱光敏感,负责照度识别,广泛分布在视网膜外围,构成全眼照度识别的基础。双极细胞处于更深层,接受上述两种细胞反馈的信号,根据光感受野的区域分成“开”、“关”两种形态。水平细胞横亘在外网状层,负责信息之间的传递,调节整体亮度并强化画面视觉边缘。

上述的人眼成像原理,为神经形态视觉传感器的研发提供了理论基础和灵感来源。通过对各种细胞的功能抽象和电子元件功能替代,我们可以设计出具有不同特性和优势的神经形态视觉传感器。
\begin{figure}[ht]
  \centering
  \includegraphics[width=\textwidth]{human_eye.png}
  \caption{人眼及其视网膜结构示意图}
  \label{fig:human_eye}
\end{figure}


\subsection{上述系统对脉冲相机的启发}
来自北京大学的Huang教授及其团队于2018年提出了仿视网膜中央凹采样模型(Fovea like Sampling Model,FSM)\cite{7923720}。此模型借鉴了上述人眼成像机制中视网膜中央凹的感知结构,突出其光敏优势,通过脉冲发放频率和脉冲宽度对光强进行积分,且由于自然界中光强客观存在,脉冲必定发放,通过人为或自适应设定脉冲发放阈值调节“光敏度”保持合理的脉冲密度,籍此避免了传统曝光相机在过暗和过亮情况下无法记录信息的问题。Huang教授及其团队进一步具象化上述原理,自主研发脉冲相机(Spike Camera),采用脉冲序列\cite{dong2018spike}形式传输数据,即高速脉冲帧。随后该团队在时间分辨率和空间分辨率上对硬件开展进一步的提升。表列出了两代Spike Camera的各项详细参数。

\begin{table}
    \centering
    \caption{Spike Camera的各项详细参数}
    \label{tab:spike_camera_parameter}
    \begin{tabular}{ccc}
      \toprule
      Spike Camera & -Gen1\cite{dong2018spike}&-Gen3\cite{Huang_Tiejun110} \\
      \midrule
      年份 &2018&2020 \\
      空间分辨率(像素数) &400\times250&1000\times1000   \\
      时间分辨率($\upmu$s) &50&25    \\
      动态范围(dB) &>110&>110 \\
      最大数据通量(eps) &2G&40G   \\
      芯片尺寸(mm\textsuperscript{2}) &10\times6&20\times20    \\
      像素尺寸(平方毫米)&20\times20&17\times17 \\
      芯片制造工艺(nm) &110&110   \\
      芯片工作电压(V) &12&12    \\
      填充系数 &13.75\%&13.75\% \\
      芯片功耗(mW) &370&3000   \\
      数据接口 &USB3.0&USB3.0    \\
      \bottomrule
    \end{tabular}
  \end{table}

\section{脉冲相机工作原理}

\begin{figure}[ht]
  \centering
  \includegraphics[width=\textwidth]{spike_camera_fig_clear.png}
  \caption{脉冲相机Gen1实例示意图}
  \label{fig:spike_camera_working_principle}
\end{figure}


脉冲相机的成像原理模拟视网膜中央凹区域神经元发放脉冲的原理,每一个像素点均由独立的电路结构构成,包含光感受器,自复位电路和脉冲读出电路,分别对应着脉冲信号产生的三个步骤——积分、复位和读出。在积分阶段感光部分的光电二极管持续吸收光子,使与其并联的电容上积累的电荷减少,电压降低。当电容上的电压降低至设定阈值$V_{th}$时比较器产生反转信号并驱动自复位电路生成复位信号复原光电二极管,使之复原到重新从零积累光子的状态,即新的积分阶段。当新的时钟信号来临时,自复位电路的信号被读出电路读取,且该信号被清除,重新进入复位前的阶段。而整个电路中可调节的参数为阈值电压和时钟频率。前者决定了单个脉冲所代表的光子数量,后者决定了芯片的最大脉冲发放频率,即芯片的时间分辨率。用数学公式表述上述过程如下。
\begin{equation}
  \label{eq:4}
    \frac{1}{C} \int_{t}^{t+\Delta t} I_s dt \geq V_{th} 
\end{equation}

其中,$V_{th} = V_{dd} - V_{ref}$为设定电压阈值,$C$为电容,$I_s$为光电流强度,$\delta t$为积分时间,值得注意的是积分时间可以超过一个时钟周期,这对应着在环境光强过暗时可能需要多个时间周期才能积累到一个脉冲。由上述原理可推知下述理想情况下(忽略噪声,场景亮度稳定,电容无电荷泄露情况)单位时间内脉冲发放数量即脉冲频率的计算公式
\begin{equation}
  \label{eq:5}
    f_s = \frac{I_s}{CV_{th}}
\end{equation}
在一定时间$T_given$内,单个像素在理想情况下发放的脉冲数量$N_{s}$如下
\begin{equation}
  \label{eq:6}
    N_{s} = T_{given} f_s = \frac{T_{given} I_s}{CV_{th}}
\end{equation}

由此可知一定时间内发放脉冲的数量和场景光强成正比。
对于相机而言,成像的动态范围是其重要参数之一,决定了其能够测量的最高/低亮度的比值。根据动态范围计算公式
\begin{equation}
  \label{eq:7}
    DR = 20 \log_{10} \frac{L^{max}}{L^{min}}
\end{equation}

其中$L_{max}$和$L_{min}$分别为相机能够测量的最高亮度和最低亮度,$DR(Dynamic Range)$为动态范围。将其带入脉冲相机中得到脉冲相机动态范围计算公式
\begin{equation}
  \label{eq:8}
    DR = 20 \log_{10} \frac{N^{max}_s}{N^{min}_s}
\end{equation}
其中$N^{max}_s$为一定时间内发射的脉冲数量的最大值,对应着最大可处理的场景光强,$N^{min}_s$为一定时间内发射的脉冲数量的最小值,对应着最小可处理的场景光强。这里采用脉冲数量进行计算,隐含着脉冲相机动态范围和接受光子的时间相关。在前述理想条件下,公式为无偏差估计,但在实际情况中,由于光子以Poisson分布随机发放,元件制成过程中的内在缺陷,电路读出过程中的读出噪声,电路运行时生热产生的热噪声,电路内部暗电流等都会影响测得脉冲。特别是由于暗电流的存在,在光强为零的条件下经过特定时间也可以产生脉冲。在光强较弱的条件下,暗电流的存在可能会导致信噪比受到较大影响,且这种影响是潜藏在脉冲内部的,不可被直接分开。

按照表\ref{tab:spike_camera_parameter}所示,采用Gen1版本相机进行数据计算,在40kHz的时钟频率下,时间窗口为1s时动态范围约为92dB,而要达到表中的大于110dB的动态范围,则时间窗口不小于7.9s。


\section{压缩感知图像重建算法研究现状}
\subsection{压缩感知理论发展概述}

\begin{figure}[ht]
  \centering
  \includegraphics[width=\textwidth]{cs_principle.png}
  \caption{压缩感知框架}
  
\end{figure}
得益于21世纪以来计算机硬件条件的大幅提升和先进的机器学习算法的快速发展,计算机视觉的现实应用日益蓬勃,其解决问题的思路是:
\begin{enumerate}
  \item 获取各种形式(特定视角,特定数量,特定环境条件,特定设备,特定数据格式等)的目标任务涉及对象的视觉信息
  \item 对上述视觉信息进行初步处理(数据对齐,裁剪,增广等),形成所需的具备特定特点的数据集
  \item 利用对应算法组合对数据集进行运算并得到特定结果
  \item 将得到的最优算法在实际场景下检验有效性和鲁棒性
\end{enumerate}


但由于现实环境的时间空间充满不可量化的无穷变化,第一步为解决特定问题而准备的数据规模在不断膨胀,如果不加以处理很可能会超出计算机硬件的处理能力,导致算法无法收敛。这就要求科研工作者在数据数量和数据质量之间做出权衡。应运而生的,数据压缩感知算法在2006年被David L. Donoho等人首次提出\citep{David_compress,An_Introduction_To_Compressive_Sampling,Extensions_of_compressed_sensing},其主要思想是突破奈奎斯特采样定理,解决传统数据必须先采样后压缩的弊端,通过在采样时即采用稀疏矩阵对原数据进行稀疏化,使得原信号或信号在某个变换域上稀疏,用向量的语言表述则是使向量中尽可能多的元素为0从而减少数据量。然后通过采样矩阵将原高维信号投影到低维空间. 压缩感知相较于传统先采样后压缩的方式而言, 可以通过稀疏矩阵和采样矩阵结合为压缩矩阵,来实现采样和压缩同时进行。这样既降低硬件设备的要求, 又节约数据在传输和运算过程中的所需带宽。这是其在数据压缩层面的主要贡献。而将其应用到图像重建上,则实现了图像信息的投射变换和降维,寻找图像分解的变换基,从而在变换中将内在特征不同的图像信息和噪声信息区别开来,DCT变换\cite{DCT},快速傅里叶变换等将图片从空间域变换至频域的算法,通过删除高频信息实现了图像本征信息的保留和压缩。这是其在图像去噪,去模糊等方向上的贡献。如图\ref{fig:FFT_IMAGE}所示,经过快速傅里叶变换后图像整体视觉信息大部得到保留,图像的浅色部分代表的高频信息的损失只产生了可以被人眼接受并在识别允许范围内的噪声。这充分体现了图片信息的冗余性。


\begin{figure}[htbp]
    \centering
    % 第一行图片
    \begin{subfigure}[b]{0.45\textwidth}
        \centering
        \includegraphics[width=\textwidth]{original_image_skimage.jpg}
        \caption{原始图像}
        \label{fig:sub1}
    \end{subfigure}
    \hfill
    \begin{subfigure}[b]{0.45\textwidth}
        \centering
        \includegraphics[width=\textwidth]{fourier_transform_skimage.jpg}
        \caption{FFT转换后的频谱}
        \label{fig:sub2}
    \end{subfigure}

    % 第二行图片
    \vskip\baselineskip
    \begin{subfigure}[b]{0.45\textwidth}
        \centering
        \includegraphics[width=\textwidth]{filtered_spectrum_skimage.jpg}
        \caption{滤去高频后的频谱}
        \label{fig:sub3}
    \end{subfigure}
    \hfill
    \begin{subfigure}[b]{0.45\textwidth}
        \centering
        \includegraphics[width=\textwidth]{inverse_fourier_transform_skimage.jpg}
        \caption{过滤后逆变换所得图像}
        \label{fig:sub4}
    \end{subfigure}
    \caption{将原始图像通过快速傅里叶变换,高频率波谱去除,再进行逆傅里叶变换得到的图像}
    \label{fig:FFT_IMAGE}
\end{figure}

% ... existing code ...
\subsection{压缩感知理论数学表述}
压缩感知用于图像重建分为三个主要模块:
\begin{enumerate}
  \item 观测矩阵的设计
  \item 信号的稀疏化表示
  \item 图像信息从稀疏代码的重构
\end{enumerate}

\begin{figure}[ht]
  \centering
  \includegraphics[width=\textwidth]{cs_principle_image.png}
  \caption{观测值,原始值和稀疏代码}
  
\end{figure}
第一步观测矩阵的设计数学表达如下
\begin{equation}
  \label{eq:9}
  y = \Phi x
\end{equation}

其中$y \in R^{M \times 1}$为压缩感知的观测值,是对真实信号的一种采样,$x \in R^{N \times 1}$为原始图像信号。这里对x进行拉平处理,将一般情况下的二维图像信号拉平为一维信号便于数学处理。$\Phi \in R^{M \times N}$是测量矩阵,其作用是对原始图像信号进行采样。在此公式中,压缩比例为$\frac{M}{N}$。且为了保证且为了突破传统的奈奎斯特采样定律,需要选择合适的测量矩阵使得 $M \ll N$。在这个阶段,压缩本身是无损的。而之所以能够突破这个采样定律,是因为这个定律是建立在周期性的连续信号等间距采样上的,且没有对原始信号进行其他域上的变换。对于图像这样非周期性的数据,奈奎斯特采样是不足以充分描述和应用的。用非等间距采样例如随即亚采样往往能更好的体现出图像各个部分信息类型、密度不均匀的性质。更进一步地,为了从远小于未知解的方程个数中还原出原图像,还应该保证RIP(Restricted Isometry Property)条件,使得信号在K稀疏的条件下可以从M个参数中得到最优解。2005年Tao等人\cite{Decoding_by_linear_programming}提出了保证最优解唯一的等效RIP条件,此证明作为压缩感知原始信号还原的可行性和质量的保证。也证明,独立同分布的随机高斯采样矩阵能够作为普适的采样矩阵。下图是不同的采样矩阵的图例,

\begin{figure}[ht]
  \centering
  \includegraphics[width=\textwidth]{random_matrix.png}
  \label{fig:random_sample_matrix}
  \caption{不同的采样矩阵}
\end{figure}

2006年Emmanuel等人\cite{Emmanuel2006Stable}证明了在已知信号稀疏性的情况下,可能凭借较采样定理所规定更少的采样数重建原信号,这一理论也是压缩感知的基石,对RIP的必要性做了进一步的阐释。2007年Bananiuk等人\cite{2008A}简化了满足RIP条件的矩阵验证,在该技术从理论走向实践的过程中贡献。


第二步信号的稀疏化表示的数学表达如下
\begin{equation}
  \label{eq:10}
  x = \Psi a
\end{equation}

其中x为公式\ref{eq:9}中的原始图像信号。在真实世界的普通摄像设备对常规物体拍摄的二维图像中,图像在空间上具备较强的相关性。且图像的像素排列表示并不满足稀疏条件,不存在大量值为0的元素,因此在这一步中通过设计合适的稀疏矩阵$\Psi \in R^{N \times N}$
将x变化为转换域上的信号$a \in R^{N \times 1}$。如果a向量中有K个非零元素,则称a向量为K稀疏向量。
此时联立\ref{eq:9}与\ref{eq:10}可得到

\begin{equation}
  \label{eq:14}
   y = \Phi x = \Phi \Psi a = \Theta a
\end{equation}
其中$\Theta = \Phi \Psi, \Theta \in R^{M \times N}$,称其为传感矩阵。此时我们可以详细的列出RIP条件。由于多篇文章定义形式不一,这里采用\cite{The_restricted_isometry_property_and_its_implications_for_compressed_sensing}中定义的RIP条件。

\begin{equation}
  \label{eq:13}
  (1-\delta_k) \Vert a \Vert_2^2 \leq \Vert \Theta a \Vert_2^2 \leq (1+\delta_k) \Vert a \Vert_2^2
\end{equation}

即每一个整数$k \in \left[1,N\right]$,定义矩阵$\Theta$的等距系数$\delta_{k}$为在所有k稀疏向量a情况下,都满足\ref{eq:13}公式的最小值。
从能量的角度考虑,$\Vert \Theta a \Vert_2^2 = \Vert y\Vert_2^2$,即为观测信号的能量,而根据公式\ref{eq:10}将原始信号分解成稀疏代码的矩阵$\Psi$是正交变换,根据Parseval变换定理正交变换不改变信号能量。所以通过引入参数$\delta_k$使得前后能量的差值满足一定的范围,即观测矩阵和标准正交矩阵的差别不能够超出一定范围。对应到向量上,RIP条件是保证k稀疏向量在分解过程之中对应不为0的分量的能量不存在剧烈的增幅或衰减。如果允许$\delta_k = 0$,则$\Vert a \Vert_2^2 = \Vert y\Vert_2^2$,二者能量完全相等,证明观测矩阵本身亦是正交变换,而按照公式\ref{eq:13}定义,$\delta_k \in (0,1)$,所以这证明$\Phi$不可能是正交矩阵,保证了$M < N$这个压缩的前提。

在2008年Bananiuk等人\cite{Compressive_sensing_[Lecture_Notes]}解决了RIP问题易于定义不便寻找的问题,提出了RIP的等价定义,即$\Phi$与$\Psi$之间彼此不相关,相关性定义如下:

\begin{equation}
  \label{eq:11}
  \mu(\phi, \psi) = \sqrt{n} \cdot \mathop{\max}_{1 \leq k,j \leq n} |<\phi_k,\psi_j>| \Rightarrow \mu(\phi, \psi) \in \left[ 1,\sqrt{n}  \right]
\end{equation}

其中$\mu(\phi, \psi)$越接近于1,则越不相关。将\ref{eq:14}转变成优化问题如下

\begin{equation}
  \label{eq:12}
  \min \Vert a \Vert_0 \quad \text{s.t.} \quad y = \Phi x = \Phi \Psi a
\end{equation}

第三步图像信息从稀疏代码的重构即是将优化所得结果a代回\ref{eq:10}则可求出原始信号x。但是考虑到$\Phi \in R^{M \times N}$ 且  $M \ll N$,那么上述优化问题是一个NP难问题,无法直接求解。需要将公式进行改造以转换为凸优化问题或者采用贪婪算法进行近似求解。2008年Emmanual等人\cite{The_restricted_isometry_property_and_its_implications_for_compressed_sensing}证明了当$\delta_{2k} < 1$时可以保证零范数问题有唯一的稀疏解,而当$\delta_{2s}<\sqrt2-1$时则可以保证0范数和1范数等价。零范数求解为NP-hard问题,在此前提下将其转化为1范数求最优化问题,由于此时问题被转换为凸优化问题,所以可以保证其局部最优解即为全局最优解。

对公式\ref{eq:12}添加不同的约束项会得到不同的目标函数。

\begin{figure}[ht]
  \centering
  \includegraphics[width=\textwidth]{different_norm.png}
  \label{fig:differnet_norm}
  \caption{不同范数约束项对优化函数的影响}
\end{figure}

% 在需要插入公式的位置添加以下代码
\begin{equation}
    \begin{aligned}
      \arg\min \Vert \Theta x - y \Vert_2^2 +& \lambda \Vert \Psi a \Vert_0  \\
      \arg\min \Vert \Theta x - y \Vert_2^2 +& \lambda \Vert \Psi a \Vert_1  \\
      \arg\min \Vert \Theta x - y \Vert_2^2 +& \lambda \Vert \Psi a \Vert_2 
    \end{aligned}
    \label{eq:different_norm}
\end{equation}

在上述公式\ref{eq:different_norm}中,$\Vert \Theta x - y \Vert_2^2$是数据保真项,使得最终结果的整体趋势是使得$\Theta x$尽可能地靠近$y$,而后面的$\lambda \Vert \Psi a \Vert_m(m\in N)$作为约束项,为整个公式添加在优化过程中保证稀疏代码的稀疏度的约束。在实践之中,不同范数的约束可形象地表示为图\ref{fig:different_norm}。特别的,针对最常用的l1范数和l2范数约束项,经实验有图\ref{fig:l1_and_l2_norm}结果,表明l2范数约束项在确保稀疏代码稀疏度上效果不如l1。这从图\ref{fig:differnet_norm}中可以看出,当数据保真项所代表的直线沿着法线方向移动时,l1范数可以确保其在与约束项代表的中央图形接触时,总能够保证唯一的交叉点,根据凸函数性质其即是全局最优点。而在l2范数中,寻找唯一切点的迭代过程是复杂而不确定的,往往会得到两个交点,而无论其中的哪一个都不是最优解。而从范数自身定义考虑,l1范数是稀疏代码所有项绝对值之和,而l2范数是稀疏代码所有项绝对值的开方,后者对于绝对值较大的项相较于前者有放大作用,使得靠近零的项的作用体现较弱,不利于收敛。
\begin{figure}[ht]
  \centering
  \includegraphics[width=\textwidth]{l1_and_l2_norm.png}
  \label{fig:l1_and_l2_norm}
  \caption{l1和l2范数对所得稀疏代码的影响}
\end{figure}

传统压缩感知算法主要有基追踪算法(BP)、内点法(IPM)、梯度投影
法(GPSR)  、 迭代阈值法(ISTA) 和交替方向
乘子法(ADMM)  等. 常见的贪婪算法有匹配追
踪算法(MP) 、 正交匹配追踪算法(OMP)  和压
缩采样匹配追踪算法(CoSaMP) 。其中对比详见表

\begin{table}
  \centering
  \caption{Spike Camera的各项详细参数}
  \label{tab:spike_camera_parameter}
  \begin{tabular}{ccc}
    \toprule
    Spike Camera & -Gen1\cite{dong2018spike}&-Gen3\cite{Huang_Tiejun110} \\
    \midrule
    年份 &2018&2020 \\
    
    \bottomrule
  \end{tabular}
\end{table}
